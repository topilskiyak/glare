%======================================================================
% Document Settings

\documentclass[a4paper, 12pt]{article}

\textwidth=440pt
\textheight=620pt
\hoffset=-43pt
\voffset=-50pt
\headsep=25pt
\footskip=40pt

%======================================================================
% Packages

% Languages
\usepackage[T2A]{fontenc}
\usepackage[utf8]{inputenc}
\usepackage[russian, english]{babel}

% Colours
\usepackage[dvipsnames]{xcolor}

% Maths
\usepackage{amsmath}

% Latex Settings
\usepackage{enumitem}
\setlist{
    nosep
}

% URLs
\usepackage{hyperref}
\hypersetup{
    colorlinks=true,
    linkcolor=blue,
    filecolor=magenta,      
    urlcolor=blue,
    citecolor=ForestGreen,
}

%======================================================================
\begin{document}
%======================================================================
% Title

\title{
GLARE
\\
\Large{Research Notes}
\\
\author{
Topilskiy Artem\\
Dep. Data Analysis, MIPT
}
}
\maketitle

%======================================================================
\section{Paper Summaries}

\subsection{Single Image Reflection Removal Using Deep Encoder-Decoder Network \cite{SIRR1802}}
\textbf{Link} \cite{SIRR1802} \url{https://arxiv.org/abs/1802.00094}
~\\\textbf{Idea} End-to-end SIRR using UNet-like CNN
~\\\textbf{Results} PSNR: 29.08 (val), 18.70 \cite{SIRRBench17}
~\\\textbf{Dataset} $\{(I, \alpha T)\} : I = \alpha T + \beta R * G * K$\\
where $T$ - transmission, $\alpha$ - transmission coefficient; $R$ - reflection, $G, K$ - gaussian blurs, simulating the defocus effect ($G$) and double reflection ($K$). $T, R$ taken in true intensity values (reversing \href{https://www.cambridgeincolour.com/tutorials/gamma-correction.htm}{$\gamma$-compression}).
~\\\textbf{Architecture} $cccccc \longrightarrow \searrow cc\searrow cc\searrow ccdd \nearrow dd \nearrow dd \nearrow \longrightarrow dddddd$ \\
$\text{Feature Extraction CNN}
\longrightarrow \text{Reflection Removal UNet}
\longrightarrow \text{Transmission Restoration CNN}$\\
All $(de)conv$ layers are non-padded, 64-filter, 5x5 (except first 2 and last 2 which are 9x9), followed by ReLU. The CNN parts each have 6 $(de)convs$. The UNet part consists of 6 $convs$ and 6 $deconvs$ arranged in 3 levels - skip connections are every 2 $convs$ on the way "down". Skip connection additions are performed on feature maps before ReLU. No pooling layers are used, convolutions are not padded.
~\\\textbf{Loss Function} $L = L_{l_2} + \lambda L_{VGG}$\\
$L_{l_2} = || F(I) - \alpha T ||^2_2$, $L_{VGG} = \sum_{i=1}^M \frac{1}{H_i W_i}|| \phi_i(F(I)) - \phi_i(\alpha T) ||^2_2$, $\phi_i$ is the feature map obtained by the i-th convolution after activation within VGG19.
~\\\textbf{Training}\\
\textit{Dataset} (66k/22k train/test)
Images: 2.3k \cite{IndoorScenes} + 2.6k \cite{HelLooks}, reflections - natural landscapes and malls. $T$ is resized 128x128, $R$ is cropped and resized 128x128. $\gamma = 2.2, \alpha \sim U[0.75, 0.8], \beta = 1 - \alpha, \sigma(G) \sim U[1,5]$, $K$ = kernel with 2 non-zero elements: $1 - \sqrt{\alpha}, \sqrt{\alpha} - \alpha$.\\
\textit{Optimization} $\lambda=10^{-3}$, epochs=150, batch=64, Adam($\eta=10^-4, \beta_1=0.9$)\\
\textit{Specs} Nvidia Titan X GPU, Tensorflow
~\\\textbf{Extra Notes} No domain shift needed - sythetic data is enough for irl\\

%======================================================================
\newpage
\section{Ideas}

\begin{itemize}
\item learn about optical flow in ML
\item search for more glare-related papers
\end{itemize}

%======================================================================
\section{Backlog}

\subsection{Things to read}
\begin{enumerate}

\item
Name: Learning to See Through Obstructions \\
Link: \url{https://arxiv.org/abs/2004.01180} \\
Note: Учимся на синтетике, теперь видео, удаляем гораздо более сложные вещи

\item
Name: PhotoScan: Taking Glare-Free Pictures of Pictures \\
Link: \url{https://ai.googleblog.com/2017/04/photoscan-taking-glare-free-pictures-of.html} \\
Note: старая статья про удаление отражений с глянца

\end{enumerate}

%======================================================================
\newpage
\bibliographystyle{IEEEtran}
\bibliography{research}

\end{document}
%======================================================================
